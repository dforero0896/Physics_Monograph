% !TeX program = lualatex
%%%%%%%%%%%%%%%%%%%%%%%%%%%%%%%%%%%%%%%%%%%%%%%%%%%
%% LaTeX book template                           %%
%% Author:  Amber Jain (http://amberj.devio.us/) %%
%% License: ISC license                          %%
%%%%%%%%%%%%%%%%%%%%%%%%%%%%%%%%%%%%%%%%%%%%%%%%%%%

\documentclass[a4paper,11pt]{book}
\usepackage[T1]{fontenc}
\usepackage[utf8]{inputenc}
\usepackage{lmodern}
%%%%%%%%%%%%%%%%%%%%%%%%%%%%%%%%%%%%%%%%%%%%%%%%%%%%%%%%%
% Source: http://en.wikibooks.org/wiki/LaTeX/Hyperlinks %
%%%%%%%%%%%%%%%%%%%%%%%%%%%%%%%%%%%%%%%%%%%%%%%%%%%%%%%%%
\usepackage{hyperref}
\usepackage{graphicx}
\usepackage[english]{babel}
\usepackage{tikz}
\usepackage{float}
\usepackage{amsmath}

\usepackage[compat=1.1.0]{tikz-feynman}


%%%%%%%%%%%%%%%%%%%%%%%%%%%%%%%%%%%%%%%%%%%%%%%%%%%%%%%%%%%%%%%%%%%%%%%%%%%%%%%%
% 'dedication' environment: To add a dedication paragraph at the start of book %
% Source: http://www.tug.org/pipermail/texhax/2010-June/015184.html            %
%%%%%%%%%%%%%%%%%%%%%%%%%%%%%%%%%%%%%%%%%%%%%%%%%%%%%%%%%%%%%%%%%%%%%%%%%%%%%%%%
\newenvironment{dedication}
{
	\cleardoublepage
	\thispagestyle{empty}
	\vspace*{\stretch{1}}
	\hfill\begin{minipage}[t]{0.66\textwidth}
		\raggedright
	}
	{
	\end{minipage}
	\vspace*{\stretch{3}}
	\clearpage
}

%%%%%%%%%%%%%%%%%%%%%%%%%%%%%%%%%%%%%%%%%%%%%%%%
% Chapter quote at the start of chapter        %
% Source: http://tex.stackexchange.com/a/53380 %
%%%%%%%%%%%%%%%%%%%%%%%%%%%%%%%%%%%%%%%%%%%%%%%%
\makeatletter
\renewcommand{\@chapapp}{}% Not necessary...
\newenvironment{chapquote}[2][2em]
{\setlength{\@tempdima}{#1}%
	\def\chapquote@author{#2}%
	\parshape 1 \@tempdima \dimexpr\textwidth-2\@tempdima\relax%
	\itshape}
{\par\normalfont\hfill--\ \chapquote@author\hspace*{\@tempdima}\par\bigskip}
\makeatother

%%%%%%%%%%%%%%%%%%%%%%%%%%%%%%%%%%%%%%%%%%%%%%%%%%%
% First page of book which contains 'stuff' like: %
%  - Book title, subtitle                         %
%  - Book author name                             %
%%%%%%%%%%%%%%%%%%%%%%%%%%%%%%%%%%%%%%%%%%%%%%%%%%%

% Book's title and subtitle
\title{\Huge \textbf{Geoneutrino oscillations approach to discriminate distributions of HPE in the Earth's mantle using the Monte Carlo Technique.}   \\ \huge Monography for Geoscientist degree \footnote{Universidad de Los Andes - Bogotá, Colombia}}
% Author
\author{\textsc{Daniel Felipe Forero Sánchez}\thanks{\url{github.com/dforero0896}}}


\begin{document}
	
	\frontmatter
	\maketitle
	
	%%%%%%%%%%%%%%%%%%%%%%%%%%%%%%%%%%%%%%%%%%%%%%%%%%%%%%%%%%%%%%%
	% Add a dedication paragraph to dedicate your book to someone %
	%%%%%%%%%%%%%%%%%%%%%%%%%%%%%%%%%%%%%%%%%%%%%%%%%%%%%%%%%%%%%%%
	\begin{dedication}
		Dedicated to Calvin and Hobbes.
	\end{dedication}
	
	%%%%%%%%%%%%%%%%%%%%%%%%%%%%%%%%%%%%%%%%%%%%%%%%%%%%%%%%%%%%%%%%%%%%%%%%
	% Auto-generated table of contents, list of figures and list of tables %
	%%%%%%%%%%%%%%%%%%%%%%%%%%%%%%%%%%%%%%%%%%%%%%%%%%%%%%%%%%%%%%%%%%%%%%%%
	\tableofcontents
	\listoffigures
	\listoftables
	
	\mainmatter
	
	%%%%%%%%%%%
	% Preface %
	%%%%%%%%%%%
	\chapter*{Preface}

	
	%%%%%%%%%%%%%%%%%%%%%%%%%%%%%%%%%%%%
	% Give credit where credit is due. %
	% Say thanks!                      %
	%%%%%%%%%%%%%%%%%%%%%%%%%%%%%%%%%%%%
	\section*{Acknowledgements}

	
	%%%%%%%%%%%%%%%%
	% NEW CHAPTER! %
	%%%%%%%%%%%%%%%%
	\chapter{Introduction}
	
	\begin{chapquote}{Author's name, \textit{Source of this quote}}
		``This is a quote and I don't know who said this.''
	\end{chapquote}
	
	\chapter{Relevance of the Study}
	
	\chapter{Distribution of Radioactive Elements}

	\section{Relative Abundances}
	\chapter{Radioactivity}
	\section{Overview}
	Radioactivity is the phenomenon in which a parent isotope turn into a daughter isotope, with different characteristics, through the emmission of a particle.\\
	The history of radioactivity goes back to 1896, when H. Becquerel discovered that a uranium sample emmited some kind of penetrating radiation similar to the X rays (discovered earlier that year). In the following years, three different types of emissions were identified: alpha ($\alpha$), beta ($\beta$) and gamma ($ \gamma$). The nature of each of these emissions was identified through the years, concluding that $\alpha$-particles correspond to $^4He$ nuclei, $\beta^\pm$-particles correspond to $e^\pm$ (electrons or positrons) and $\gamma$-particles are nothing but high energy photons.\\
	Rutherford discovered that radioactive phenomena was linked directly with the nucleus ($\sim 1 fm$) of a given isotope, thus, it is an entirely quantum phenomenon.\\
	The overall radioactive phenomena are described by a rather simple mathematical approach given the statistical behavior of the phenomena (which is a result, and evidence of how ``quantum''it is). Given $N$ parent isotopes in a sample at a given time $t$ and assuming no more are added, the rate of decay ($dN/dt$) is proportional to $N$ \cite{Krane1988} 
	\begin{equation}
	\frac{dN}{dt}=-\lambda N
	\end{equation}
	That, upon integration becomes 
	\begin{equation}
	N(t)=N_0\exp(-\lambda t)
	\end{equation}
	Where $N_0$ corresponds to the number of parent isotopes at time $t=0$ and $\lambda$ is called the decay constant and is unique for every isotope.\\
	This simple model has been key to the use of the radioactive isotopes to, for example, date rocks or organisms.\\
	\section{Types of radioactive decay}
	Let us now go deeper into the theory of the radioactivity.
	
	\subsection{Alpha decay}
	The theory of the alpha decay is rather simple. $\alpha$-particles are confined in a finite potential well (the nucleus $X$ ) and will, as expected, have a certain probability of tunneling said barrier. When this happens, the $\alpha$-particle will escape and leave a ``new'' (daughter) nucleus $X'$ \cite{Krane1988}. The process may be written as
	\begin{equation}
^A_ZX_N\rightarrow^{A-4}_{Z-2}X'_{N-2} + \alpha
	\end{equation}
	Where $A$ is the atomic mass number, $Z$ is the atomic number and $N$ is the charge.\\
	This decay involves both strong (nuclear) interactions and electromagnetic interactions since the potential well is given by confinement due to strong interaction between nucleons and the barrier has a decaying-exponential side, given by the Coulomb potential.
	\subsection{Gamma decay}
	Gamma decay is similar to the common electromagnetic emission due to atomic transitions, in fact, it is produced when a metastable state of an isotope decays into a more stable state through the emission of high energy photons. It should be noted that, in this case, no change in $A$ or $N$ is produced. These metastable states are common daughter isotopes to $\alpha$ and $\beta$ decays.\\
	Lifetimes for this kind of process is generally short, taking only fractions of a second to decay while some $\alpha$-decaying isotopes may have half-lives of the order of $10^3yr$.
	\subsection{Beta decay}
	This type of decay can be considered to be more complex than the other two, as it responds to more complicated undelying physics: the weak interaction. It consists in a set of semileptonic processes (that involve both leptons and hadrons) that will be described below.
	\begin{itemize}
		\item \textbf{Positive Beta decay:} 
		\begin{equation}
			p\rightarrow n + e^+ + \nu_e
			\label{eq:betaplus}
		\end{equation}
		
		\begin{figure}[H]
			\centering
			\label{fig:feynbetaplus}
		\feynmandiagram	[layered layout,horizontal=a to b] {
			a [particle	=$p$] -- [fermion] b -- [fermion] f1 [particle =$n$],
			b -- [boson,edge label'	=$W^{+}$] c,c -- [fermion] f2 [particle =$ \nu_{e}$],
			c -- [anti fermion] f3 [particle	=\(e^{+}\)],
			};		
			\caption{Feynman diagram for $\beta^+$ process.}	
		\end{figure}
		\item \textbf{Negative Beta decay:}
		
			\begin{equation}
			n\rightarrow p + e^- + \bar{\nu}_e
			\label{eq:betaminus}
			\end{equation}
		
		\begin{figure}[H]
			\centering
			\label{fig:feynbetaminus}
			\feynmandiagram	[layered layout,horizontal=a to b] {
				a [particle	=$n$] -- [fermion] b -- [fermion] f1 [particle =$p$],
				b -- [boson,edge label'	=$W^{-}$] c,c -- [anti fermion] f2 [particle =$ \bar{\nu}_{e}$],
				c -- [fermion] f3 [particle	=$e^{-}$],
			};			
			\caption{Feynman diagram for $\beta^-$ process.}
		\end{figure}

\item \textbf{Electron capture:}
\begin{equation}
p + e^- \rightarrow n + \nu_e
\label{eq:electroncapture}
\end{equation}
	\begin{figure}[H]
		\centering
		\label{fig:feynbetaminus}
		\feynmandiagram [vertical=a to b] {
			i1 [particle=$n$] -- [anti fermion] a -- [anti fermion] i2 [particle=$p$],
			a -- [boson, edge label=$W^+$] b,
			f1 [particle=$\nu_e$] -- [anti fermion] b -- [anti fermion] f2 [particle=$e^{-}$],
		};		
		\caption{Feynman diagram for electron capture.}
	\end{figure}
	\end{itemize}
	Note that in all these processes new particles are created in the nucleus, which is a consequence of the weak interaction that rules them.\\
	The particles labelled with $\nu_e$ are electron neutrinos and were proposed by Pauli in order to solve the conundrum concerning the apparently continuous energy spectra of $\beta$-decay which is against the core of quantum mechanics, the inclusion of these particles in the energy spectrum of the decay made it discrete, thus, giving a solution to the problem.\\
	For this project, we are strictly interested in the $\beta^-$-decay since it is the one that produces antineutrinos.



	
	\section{Decay Chains}
	Different isotopes in the Earth have decay chains involve $\beta-$-decay at some point, but there are a few ones that dominate the antineutrino production. 
	\huge WAITING TO WRITE SOMETHING MORE ON RELATIVE ABUNDANCES SECTION
	
	\section{(Anti)Neutrino emission}
	\chapter{Neutrino Physics}
	\normalsize
	This chapter is only an overview of the special physics that surrounds these particles, for a thorough explanation you should see reference \cite{Giunti2010} and for details on time evolution reference \cite{Ohlsson2001}.\\
	Neutrinos are neutral leptons that interact weakly and are always produced in a given  flavor eigenstate $|\nu_\alpha\rangle$ with $\alpha = e, \mu, \tau$. In equations \ref{eq:betaplus} to \ref{eq:electroncapture}, all neutrinos are labelled $\nu_e$ that stands for electron neutrino, this is because they are all associated with the ``ordinary'' matter which is composed by electrons, not tauons or muons.\\
	So, neutrinos are produced in flavor states, but these are not mass (energy) eigenstates and, since the hamiltonian entirely controls the time evolution of a quantum system, the neutrinos will not evolve (travel) in these states but will rather oscillate between the real mass eigenstates which, as the name suggests, do have a definite (and unknown) mass, while flavor states are just a linear combination of them. This may be written as
	\begin{equation}
	|\nu_\alpha\rangle =\sum_{a=1}^{3} U^*_{\alpha a}|\nu_a\rangle
	\label{eq:tansform}
	\end{equation}
	Here, we have labelled $|\nu_a\rangle$ the mass eigenstates, with $a=1, 2, 3$. $U^*_{\alpha a}$ is the matrix element of the transformation matrix $U^*$ \cite{Ohlsson2001}.\\
	The matrix $U^*$ is the complex conjugate of matrix $U$, called Pontecorvo–Maki–Nakagawa–Sakata (PMNS) matrix, that rules the mixing of the neutrinos and can be parametrized as shown in \cite{Ohlsson2001}
	\begin{equation}
U=	\begin{bmatrix}
	C_2C_3 & S_3C_2 & S_2  \\
	-S_3C_1-S_1S_2C_3 & C_1C_3-S_1S_2S_3 & S_1C_2 \\
	S_1S_3-S_2C_1C_3 & -S_1C_3-S_2S_3C_1 & C_1C_2 
	\end{bmatrix}
	\end{equation}
	
	Where $S_i\equiv\sin(\theta_i)$ and $C_i\equiv\cos(\theta_i)$. $\theta_i$ are the vacuum mixing angles. Here we have assumed there is no charge-parity (CP) phase, thus, $U=U^*$.
	\section{Time evolution}
	According to the principles of quantum mechanics, the hamiltonian of a system described by the ket $|\psi\rangle$ does define its time evolution according to
	\begin{equation}
	|\psi(t)\rangle = \exp(-i\hat{H}(t-t_0))|\psi(t_0)\rangle
	\label{eq:gentimeevol}
	\end{equation}
	In the case of the neutrinos, we have settled that they do not travel in flavor state but in mass states, then, we can write
	\begin{equation}
	|\nu_a(t)\rangle = \exp(-\hat{H}_m(t-t_0))|\nu_a(t_0)\rangle
	\label{eq:timeevol}
	\end{equation}
	In this base, the hamiltonian will be diagonal
		\begin{equation}
		\hat{H}_{m, unperturbed}= \begin{bmatrix}
		E_1 & 0 & 0  \\
		0 & E_2 & 0 \\
		0 & 0 & E_3 
		\end{bmatrix}
		\label{eq:unphammass}
		\end{equation}
	In matter, the hamiltonian of the system is perturbed in the following way
	\begin{equation}
	\hat{V}_f= A	\begin{bmatrix}
	1 & 0 & 0  \\
	0 & 0 & 0 \\
	0 & 0 & 0 
	\end{bmatrix}
	\label{eq:perturbation}
	\end{equation}
	Where $A$ depends on the density of the matter the neutrino is going through, and is defined as 
	\begin{equation}
	A(r)\approx\pm \sqrt{2}G_F\frac{\rho(r)}{m_N}
	\label{eq:A}
	\end{equation}
	Here, $G_F$ is the Fermi coupling constant, $\rho(r)$ is the matter density and $m_N$ is the nucleon mass.\\
	
	The perturbation $V_f$ ($f$ stands for flavor base) only affects electron-flavored neutrinos because matter, as we see it, is electronic. Taking this into account, the perturbed hamiltonian, in mass basis is
		\begin{equation}
		\hat{H}_{m}= \begin{bmatrix}
		E_1 & 0 & 0  \\
		0 & E_2 & 0 \\
		0 & 0 & E_3 
		\end{bmatrix} + U^{-1}V_fU
		\label{eq:hammass}
		\end{equation}
	
	Introducing equation \ref{eq:hammass} into \ref{eq:timeevol} (this involves some complex algebra described in \cite{Ohlsson2001}) we can get the time evolution operator $\hat{U}_f$. Then, the transition probability from an initial $\alpha$ state to a $\beta$ one will be
	\begin{equation}
	P_{\alpha\rightarrow\beta}=|\langle\beta|\hat{U}_f|\alpha\rangle|^2
	\label{eq:transprob}
	\end{equation}
	This will be crucial on this project, as the neutrino oscillation phenomena will affect the detection, and the path that the neutrino follows defines how much it will be affected, therefore, the effect of the phenomena is not trivial at all.\\
	
	\chapter{The Model}
	\chapter{Results}
	\chapter{Analysis}
	\chapter{Further Problems}
	\chapter{Conclusions}
	\bibliographystyle{unsrt}
	\bibliography{refs}
		
		
	
\end{document}
